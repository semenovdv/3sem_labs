\section{Тест производительности}
\begin{lstlisting}[language=C]
clock_t start_my = clock();
 /* some work */
clock_t end_my = clock();
std::cout <<"map " << (double)(-(start_my - end_my)) / CLOCKS_PER_SEC << std::endl;
\end{lstlisting}
Первый результат - результат карманной сортировки. Второй - сортировки вставкой.

\begin{center}
\begin{table}[h]

\label{tabular:timesandtenses}

\begin{tabular}{c|c|c}
Кол-во тестов & map & RBTree\\

100 & 0.006953 & 0.00136\\
1000 & 0.023991 & 0.015864\\
10000 & 0.311768 & 0.12934\\
\end{tabular}
\end{table}
\end{center}
Как мы видим, наше красно-черное дерево работает гораздо быстрее.


 
\pagebreak
