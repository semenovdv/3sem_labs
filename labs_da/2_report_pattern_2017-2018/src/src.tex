\section{Описание}
Программа должна обрабатывать строки входного файла до его окончания. Каждая строка может иметь следующий формат:
\begin{enumerate}
\item + word 34 - добавить слово «word» с номером 34 в словарь.
Программа должна вывести строку «OK», если операция прошла
успешно, «Exist», если слово уже находится в словаре.
\item - word — удалить слово «word» из словаря. Программа должна
вывести «OK», если слово существовало и было удалено,
«NoSuchWord», если слово в словаре не было найдено.
\item word — найти в словаре слово «word». Программа должна вывести
«OK: 34», если слово было найдено; число, которое следует за
«OK:» — номер, присвоенный слову при добавлении. В случае,
если слово в словаре не было обнаружено, нужно вывести строку
«NoSuchWord».
\item ! Save /path/to/file — сохранить словарь в бинарном компактном
представлении на диск в файл, указанный параметром команды.
В случае успеха, программа должна вывести «OK», в случае
неудачи выполнения операции, программа должна вывести
описание ошибки (см. ниже).
\item ! Load /path/to/file — загрузить словарь из файла. Предполагается,
что файл был ранее подготовлен при помощи команды Save. В
случае успеха, программа должна вывести строку «OК», а
загруженный словарь должен заменить текущий (с которым
происходит работа); в случае неуспеха, должна быть выведена
диагностика, а рабочий словарь должен остаться без изменений.
Кроме системных ошибок, программа должна корректно
обрабатывать случаи несовпадения формата указанного файла и
представления данных словаря во внешнем файле.

\end{enumerate}


Для всех операций, в случае возникновения системной ошибки
(нехватка памяти, отсутствие прав на запись и т.п.), программа
должна вывести строку, начинающуюуся с «ERROR:» и описывающую
на английском языке возникшую ошибку.
\pagebreak

\section{Исходный код}
Основные этапы написания кода:
\begin{enumerate}
\item Считывание исходных данных
\item Определение операции 
\item Выполнение операции
\item Вывод результата
\end{enumerate}

Для хранения значений мы используем класс Node, который и является элементом дерева. В нем мы определяем $unsigned long long mvalue$  и $char *mkey$, где и храним значения.

После вставки элемента в само дерева, вызывается процедура балансировки, которая перестраивает дерево в соответствии с правилами построения красно-черного дерева.

При удалении вызывается аналогичная процедура. 

Балансировка достигается за счет использования функций поворота узлов дерева.


\begin{lstlisting}[language=C]


typedef enum { BLACK, RED } nodeColor;

class Node {
    public:
        nodeColor color; // enumed color
        Node *left;
        Node *right;
        Node *parent;
        unsigned long long m_value = 0;
        char *m_key = nullptr;

        Node(char *key, unsigned long long value){
            int i = 0;
            if(key != nullptr){
                int len_key = 0;
                len_key = strlen(key);
                m_key = (char*)malloc((len_key+1) * sizeof(char));
                for(i = 0; i < len_key; i++){
                    m_key[i] = key[i];
                }
            }
            m_key[i] = '\0'; // valgrind lenkey+1        
            m_value = value;
            color = RED;
        };
        
        Node(nodeColor color, Node *left, Node *right, 
            Node *parent,  char *key, unsigned long long value){
            int i = 0;
            if(key != nullptr){
                int len_key = 0;
                len_key = strlen(key);
                m_key = (char*)malloc((len_key+1) * sizeof(char));
                for(i = 0; i < len_key; i++){
                    m_key[i] = key[i];
                }
            }
            m_key[i] = '\0'; // valgrind lenkey+1        
            m_value = value;
            this->color = color;
            this->left = left;
            this->right = right;
            this->parent = parent;
        };
        
        Node(){ 
        };
        ~Node(){
            free(m_key); // valgrind  1- to free, 2 - in deletehidden

        };
    };

void TRBTree::RotateLeft(Node *node) {

    Node *right_child = node->right; 
    node->right = right_child->left;

    if (right_child->left != nil)
        right_child->left->parent = node;
    
    if (right_child != nil) right_child->parent = node->parent;

    if (node->parent != nil) {
        if (node == node->parent->left)
            node->parent->left = right_child;
        else 
            node->parent->right = right_child;
    } else {
        root = right_child;
    }

    right_child->left = node;
    if (node != nil) node->parent = right_child;
    
}


void TRBTree::FixInsertRBTree(Node *node) {
    
    while (node != root 
        && node->parent->color == RED) 
    {
     
        
        if (node->parent == node->parent->parent->left) {
            
            if (node->parent->parent->right->color == RED) { 
                node->parent->parent->color =  RED;
                node->parent->parent->right->color =  BLACK;
                node->parent->parent->left->color =  BLACK; 
                node = node->parent->parent; 
            } else {
               
                if (node == node->parent->right){ 
                    node = node->parent;
                    RotateLeft(node);
                } 
                node->parent->color =  BLACK;
                node->parent->parent->color =  RED;
                RotateRight(node->parent->parent); 
            }
        } else {
            
            if (node->parent->parent->left->color == RED) {
                node->parent->parent->color = RED;
                node->parent->parent->right->color = BLACK;
                node->parent->parent->left->color = BLACK; 
                node = node->parent->parent;
            } else {
                if (node == node->parent->left){ 
                    node = node->parent;
                    RotateRight(node); 
                } 
                node->parent->color =  BLACK;
                node->parent->parent->color =  RED;
                RotateLeft(node->parent->parent);
            }
        }
    }

    this->root->color =  BLACK;
}
\end{lstlisting}


\pagebreak

\section{Консоль}
\begin{alltt}
denis@denis-Extensa-2510G:~/labs_3sem/labs_da/2lab_debug/2lab ./2lab < tests/01.t > result.txt

denis@denis-Extensa-2510G:~/labs_3sem/labs_da/2lab_debug/2lab diff tests/01.a result.txt 
\end{alltt}

\textbf{Пример теста 01.t:}
QPXLHFZWLTFXXZNMZGXSTNJUYKCQIAXRHDVQMSXYOHDIYCRLVPDCHHYOPYBIRDUPKXJRVVESDYWCPHTKOLYMELHLMOKAHQJHFAHGSPMZRJVHQNAOEKMOGPHYTGOROLKQQMZWEMOVMSULJGHJUCBEMHXJDORXZGDUKIXECRTPXLLICGRCPVCJDQSYDWLQFKVGTGKXDHDGWREHFZDGUQOXWWDVNDKOIXRJVVEKFPVHXLVJPSQSNVKNUQIVXKNIZXQM

! Save test

EEZCKZGUEGADTBILUUFUWTDAGUNRYWUUDSHLQRQCDSTUBVEITSPTNGSGAZCODBUHKIYSFSOWVTRWVFOZNNDLLEJVVJQDAJTTUVPXWKRSVUCWEQLOZAOJOIQROUXPTRRWHEDYWAUKTTHRTEUWFFFCADHAZSBRQQBXASPKWYKXPBTNOSLEDADJAJMHEZMZSEZHOYBQEPDXNDTHMAAMRLBXRITWENSETMTJDQCBCDRSTVSHKHNLSMPQWLNIBRBQHOBV
+ MKNULGHFEJSNDKRIAPYEVGWEWWPQPWOGYDNKUNSMGRCKDTHATELLKVOCQTKLBXAAKXZYUANDREBQQWJJQDZBUWKKDMPKIYBXICZFJOOTWESABIACHUBMZHYAWXIFZLJZGIBKQTFVDYVOFDLWWTZAFSMQQFHYHVWDYANGTGGLQKKSRIKUTUSGYJZYKQXXFHPUXSUOSNOHFZDRXEKNULUJYRAASJLJTTGMIHCLDLEEUJJGVAYCCMFMEKCAZOQNRGNV 3784958133452695920

- QMLDOAXVYZXBXCOBGBOCETJBGDBNYYSOGRDRXDQLQDBLDKINUCPCKSMJUEWWSRFRNEQRSWKBSVUOBUJGCYPGHSYQOLEWWQEVZWRPZNUYUVXSRVWKOOHNDADIJDNBQLXNACDJFTXKPTZGIFSYJUOWCBYOLIBLFIAMDLDBHKJJBXSMIZTZAFRYVHFQIFXOWHZGCNKPHQAGXEGZNPGIWJEDBQVXVOADPQOQSDPNSUHUWTZXWRGYHYZHVLWRVPHQPXJT

+ CHTVMOEKLWTESHKEOOPFWOOZGAPCSDLBEASIBZOCNAKVQHXZWYZEYNGHDFWLAJFVQLWNTBRJWLODMIZRXXSAKECXHBOITLFZYRMHFRATZVAKOTXFNMFFEREAJIEIEEKXUFTGOGAPHIXZQWCDADLESEUDFVVAGEVSZDKWCTGAAZTQMCYWLEQNRLVRTIVERNXIKAIKZPLJCWXONCSKATAGJTZGSFWXYIVBGGNHHHBXOHQBGBRRFBCHLUWXHXPRJQYM 86137190829105695

- ZTEQEQCAMHOBQXUTRWXKEGXTLEVTMVULAEWWRPKCNZXJSGEUGNHEFWNVGXNNTQDKIRGQDLINNKHVMDPLBUZQYKUVKFTTNTYEOOYBUOSLRVGLGWQQTIXEVGQVYVQDDLKTGIVAXTZNFYNDSRRHGTPHCIIGJWHNFICHPGEPSZEVIXFGADDTSUUTUXOVOINVWKAXFIQBSOAHDLYXAFPWRXRGTYAZCRPRSYHWJYUTDMWDQFHXYYEEJHYAQQLGJMLWAXJU

- MACYDKBVCXWLPBJWFJGQHOGEQZOSFYWIIGBMCMICAHPFQLSDUSJVFYZWVCUVFIZCUOSDKLBJUMRUTCTAPLUHBTOYDPPBJJJPUQCREQQDTRYRKCXVJNMUUHEYCKGJNEMVSNQTEWAKWMPSJIHULVLROSRUPCOAJGFJFONLCZHABKLKFTTZJJSMTDUFVKMGSESZSJXGWTYLAVUQRRMIDXAQBWALILJGPUWSWLWPDTEMFRIQFOPXSNWSDECETKHXBEWK

chtvmoeklwteshkeoopfwoozgapcsdlbeasibzocnakvqhxzwyzeynghdfwlajfvqlwntbrjwlodmizrxxsakecxhboitlfzyrmhfratzvakotxfnmffereajieieekxuftgogaphixzqwcdadleseudfvvagevszdkwctgaaztqmcywleqnrlvrtivernxikaikzpljcwxoncskatagjtzgsfwxyivbggnhhhbxohqbgbrrfbchluwxhxprjqym

! Save test

\textbf{Полученный ответ 01.a:}

NoSuchWord

OK

NoSuchWord

OK

NoSuchWord

OK

NoSuchWord

NoSuchWord

OK: 86137190829105695

OK

\pagebreak

