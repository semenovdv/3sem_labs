\section{Описание}

Нужно составить программу, которая находит в тексте заданный образец с помощью алгоритма Бойера-Мура.


{\bfseries Формат входных данных:} Искомый образец задается на первой строке входного файла. Затем следует текст, состоящий из слов, в котором нужно найти заданный образец. Никаких ограничений на длину строк, равно как на количество слов в них, ненакладывается. 


{\bfseries Формат выходных данных:} В выходной файл нужно вывести информацию о всех вхождениях искомого образца в
обрабатываемый текст: по одному вхождению на строчку. Следует вывести два числа через запятую: номер строки и номер слова в строке, с которого начинается найденный образец. Нумерация начинается с единицы. Номер строки в тексте должен отсчитываться от его реального начала (то есть, без учёта строк, занятых образцами).Порядок следования вхождений образцов несущественен.


Алгоритм Бойера Мура основан на двух правилах: поиск шаблона ведется слева направо, а сравнивание шаблона с подстрокой идет справа налево. В случае несовпадения (или, наоборот, полного попадания) используются две заранее вычисляемые функции: функция плохого символа и функция хорошего суффикса.

\pagebreak

\section{Исходный код}


Сначала мы считываем искомую строку и вычисляем с помощью нее все необходимые функции (плохого символа $map<char, int> bad\_char$ и хорошего суффикса $vector<int> good\_suffix$). Выполняется это все в конструкторе класса для более удобной работы. Особое внимание следует уделить правильному считыванию литер. Для организации хранения позиций слов я использовал очередь, состоящую из пар $<$номер строки, номер слова$>$. 

Затем используем простой алгоритм обработки данных:
\begin{enumerate}
\item{Будем считывать и подготавливать текст, пока его размер не будет равен шаблону}
\item{Cравним шаблон и подстроку}
\item{Cдвинем на нужное количество позиций и удалим ненужные символы}
\item{Cнова считаем текст и подготовим его}
\end{enumerate}
	


\begin{lstlisting}[language=C]
void work2(){
       
        read_text();
        

        do {
            prepare_text();
            
            if (textsize < pattsize) {
                read_text();
                //prepare_text();
            }
            //make_text();
            
            compare();
            moving();
            
            prepare_text(); // CHANGED
            
            read_text();
            

        } while (textsize == pattsize);
    
    }

 void compare(){
        int i;
        
        for (i = pattsize-1; i >=0 && pattern.pattern[i] == text[i]; i--);
         
        if (i < 0) {
            pair<int, int> mytmp = pos.front();
            
            cout << mytmp.first << ", " << mytmp.second << endl;
            move = pattern.good_suffix[0];

        } else {
            map<char,int>::iterator it;
            it = pattern.bad_char.find(text[i]);
            if (it == pattern.bad_char.end())  {
                move = pattsize-1;
            } else {
                move = it->second;
            }

            if (pattern.good_suffix[i+1] > (int)(move)) {
                move = pattern.good_suffix[i];
            } 
        }
    }
    
void prepare_text() {
        
        if (text[0] == ' '){
            if (char_before_all != ' ') {
                pos.pop();
            }
            text.erase(0, 1);
            textsize--;         
        }    
    }
    
\end{lstlisting}


\pagebreak

\section{Консоль}
\begin{alltt}
denis@denis-Extensa-2510G:~/labs_3sem/labs_da/4lab_debug\$ ./4lab < test5.txt 

1, 16

3, 12

11, 3

13, 26

denis@denis-Extensa-2510G:~/labs_3sem/labs_da/4lab_debug\$ ./4lab < test3.txt 

1, 2

4, 1

4, 3

\end{alltt}
\textbf{Пример теста 1:}

he llo he

he  
      
    llo      
      
he    llo   he llo   he

\textbf{Полученный ответ:}

1, 1

3, 1

3, 3

\textbf{Пример теста 2:}

a a b a a c a a b a a b a

a a b a a b

a

a c a a b a a b a a b a a c

a a b a a b a a b

a a b a a b

\textbf{Полученный ответ:}

1, 4

4, 9

\pagebreak

