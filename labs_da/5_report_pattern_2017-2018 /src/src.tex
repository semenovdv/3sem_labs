\section{Описание}
Программа должна обрабатывать две строки входного файла, строить по ним обобщенное суффиксное дерево и находить в нем наибольшую общую подстроку двух строк.

{\bfseries Входные данные:} две строки.

{\bfseries Выходные данные:} на первой строке нужно распечатать длину максимальной общей подстроки, затем перечислить все возможные варианты общих подстрок этой длины в порядке лексикографического возрастания без повторов.

\pagebreak

\section{Исходный код}
Основные этапы написания кода:
\begin{enumerate}
\item Анализ поставленной задачи
\item Сбор необходимой информации для реализации
\item Написание кода:
\begin{enumerate}
\item Определение классов и функций
\item Написание конструкторов
\item Написание процедур добавления суффикса в дерево
\item Реализация обхода дерева в глубину
\end{enumerate}
\item Тестирование кода, отладка
\item Составление отчета
\end{enumerate}


Для хранения подстрок мы создадим класс TSTNode, каждый элемент которого будет содержать $map<char, TSTNode*>$ содержащий следующие переходы и два числа: $begin$ и $end$  для экономии места. 

Основная процедура построения - это PushBack.
Я использовал алгоритм Укконена. Чтобы построить дерево мы должны поочередно вызвать процедуру, передавая ей число от $0$ до кол-ва литер $- 1$. Процедура сама найдет нужные строки в веденном тексте.

Вторая важная функция - это поиск наибольшей общей строки в дереве. По мере своего продвижения она помещает пройденные ноды в deque, чтобы затем можно было легко восстановить строки, скрытые за абстракцией дерева.


\begin{lstlisting}[language=C]

void TSTree::PushBack1(int pos_ch) {
    how_much_left++;
    char new_ch = text1[pos_ch]; // char to be pushed
       suffix_link_change = root; 

    while(how_much_left) {
        if (active_length == 0) {
            active_char_pos = pos_ch;
        }
        std::map<char, TSTNode* >::iterator pos = active_node->next_nodes.find(text1[active_char_pos]);
         
        TSTNode *next = nullptr;
        if (pos == active_node->next_nodes.end()) {
            TSTNode *leaf = new TSTNode(pos_ch, textsize1 - pos_ch, true,false); 
            active_node->next_nodes[text1[active_char_pos]] = leaf;
            add_link(active_node);
        // we found link, 2 ways    
        } else {
            next = pos->second;
            // if we can go dipper 
            if (active_length >= next->length) {
                active_char_pos += next->length;
                active_length -= next->length;
                active_node = next;
                continue;
            }
            // 3 rule, char (prolonged) nalready in node, have to know nex, false, truet
            
            if( text1[next->begin + active_length] == new_ch) { 
                ++active_length;
                add_link(active_node);
                break;
            }
            
            TSTNode *split = new TSTNode(next->begin, active_length, true, false);
            TSTNode *leaf = new TSTNode(pos_ch, textsize1-pos_ch, true, false);
            
            active_node->next_nodes[text1[active_char_pos]] = split;
            
            split->next_nodes[new_ch] = leaf;
            next->begin += active_length;
            next->length -= active_length;

            split->next_nodes[text1[next->begin]] = next;
           

            add_link(split);
        }

        how_much_left--;
        if(active_node == root && active_length != 0) {
            // going to previous suffix
            active_length--;
            active_char_pos = pos_ch - how_much_left + 1;            
        } else {
            active_node = (active_node->slink) ? active_node->slink : root;
        }
    }
}

int TSTree::FindLCS_Node(TSTNode *node, size_t depth) { // ??? &str

    deq.push_back(node);
    
    int got = 5;
    int to_ret = 5;
    depth = depth + node->length;

    for (std::map<char, TSTNode *>::iterator it = node->next_nodes.begin(); it != node->next_nodes.end(); ++it) {
        got = FindLCS_Node(it->second, depth); // never got 5

        if (got == 0) to_ret = 0;
        if (got == 1) {
            if (to_ret == 5) to_ret = 1;
            if (to_ret == 2) to_ret = 0;
        } 
        if (got == 2){
            if (to_ret == 1) to_ret = 0;
            if (to_ret == 5) to_ret = 2;
        }
    }

    if (to_ret == 5){
        if (node->is_s1) to_ret = 1;
        if (node->is_s2) to_ret = 2;
    }

    if (to_ret == 0) {  
        if (depth > out.first) {
            out.first = depth;
            out.second.clear();
            out.second.insert(GetStr());
        } 
        if (depth == out.first) {
            out.second.insert(GetStr());
        }
    } 
    
    deq.pop_back();

    return to_ret;
    
}

\end{lstlisting}


\pagebreak

\section{Консоль}
\begin{alltt}
denis@denis-Extensa-2510G:~/labs_3sem/labs_da/5lab_release/5lab\$ ./5lab 
hellohohlandi
hohhel
3
hel
hoh

denis@denis-Extensa-2510G:~/labs_3sem/labs_da/5lab_release/5lab\$ ./5lab < text1.txt 
998
thtprytrtvuttqtrvtquthtpvtturtuhtrrhtqyxrprthurrpytxurtrxtqyxrprthurrpytrpyrxvhtpw
prtrhpvxvpvtrthtwrqturttrhtrthtrpyrxvhtpwuryurrutryqturtwpxvrrtxtrxqutxvthxrpythtr
prytrtvuttqtrvtqthxhtptrhuqtthtuxrtthxvttwhtvxtwxvthxprytrtvuttqtrvuxtptpttrtvtxtt
rhpvtrtxtththtptrwxthutwrxtttptrxxptptrtpthttvppprxtpthtrxurptxpqutthttqpprytrtvut
tqtrvptthtqttuthxuxtxrutxxprtptxurptxpqutyurptrxuxrrxvhtprttrxrtxxhwthtuxtpyqtutyu
rppuxutpquthwtptpptxtprytrtvuttqtrvphwtvttxvvtwtrttthtwruurttppxvpxpttrtrxrttxttqr
tppqtqyqthhuppqyrputtrxrtthtttqwtrtprtpprtqythupuvutttrtxttthtpvtturtuhtrrhtputhrx
rprthurrpytrttptpttprrhtqtrtrttyuppttvtqtrtxtxpvupvttvxhrhprprttrtttrxvprxxtprtuth
tprytrtvuttqtrvtqthtpvtturtuhtrrhtpxtxptxtxvqyyttttthtpvtturtuhtrrhtqyxrprthurrpyt
rtttxprppxqhtxpxxthtrthtpttpvutxxxprptuxttxtyxvthtqrqtvptyyttryvthtuxvtrpvtpxpvxth
tpwxththttwxttxpvxxthtpvtturtuthtqutrprqurtvxxxthtpvtturtuthtptrtqpxxthtpvtturtuth
ttvxttrthuqxthtpvtturtuthtqtqprhtrxxthtpvtturtuthtqtryrrttxxxthtpvtturtuthtrpptrqt
trhtpvtturtxpr

\end{alltt}


\pagebreak

