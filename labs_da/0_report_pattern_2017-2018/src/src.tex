\section{Описание}
Требуется разработать программу, осуществляющую ввод пар «ключ-значение», их упорядочивание по возрастанию ключа алгоритмом Карманной сортировки (Bucket Sort) за линейное время и вывод отсортированной последовательности. 

Как сказано в \cite{Kormen}: \enquote{В случае Карманной Сортировки (bucket sort) предполагется, что входные данные подчиняются равномерному закону распределения. Время работы в среднем случае при этом оказывается равном О(n).  При карманной сортировке предполагается, что входные числа генерируются случайным процессом и равномерно распределены в интервале $[0,1)$.

Карманная сортировка разбивает интервал $[0,1)$ на $n$ одинаковых интервалов, или карманов (buckets), а затем распределяет $n$ входных чисел. Поскольку последние равномерно распределены в интервале $[0,1)$, мы предполагем, что в каждой из карманов попадает не очень много элементов. Чтобы получить выходную последовательность, нужно просто выполнить сортировку чисел в каждом кармане, а затем последовательно перечислить элементы каждого кармана. 

При составлении кода карманной сортировки предполагается, что на вход подается массив $А$, состоящий из $n$ элементов, и что величина каждого принадлежащего массиву элемента $А[i]$ удовлетворяет неравенству $0 =< A[i] < 1$. Для работы нам понадобится вспомогательный массив связанных списков (карманов) $B[0..n-1]$}.

\pagebreak

\section{Исходный код}
Основные этапы написания кода:
\begin{enumerate}
\item Считывание исходных данных
\item Разделение по карманам
\item Сортировка каждого отдельного кармана
\item Вывод результата
\end{enumerate}

Мы храним ключ и значение в структуре KeyValue, реализованной в "KeyValue.h".

Во время считывание вызываем конструктор KeyValue() и кладем получившийся объект в вектор.

Для распределения всех чисел в полуинтервале $[0;1)$, мы делим ключи на максимально возможное значения ключа + 10, чтобы не получить значение $1$


\begin{lstlisting}[language=C]
#include "vector.cpp"
#include "KeyValue.h"
#include <cmath>
#include <limits.h>
#include <iostream>      

int main(){
    std::cout.precision(20); 
    long double key;
    std::string value;
    NVector::TVector<KeyValue> input;
	
	//reading structure
    while(std::cin >> key >> value){
        if(input.Push_back(KeyValue(key, value)) == false){
            std::cout << "ERROR push_back" << std::endl;
            return 0;
        }
    }

    const long double  num_of_el = input.Size();

    NVector::TVector<NVector::TVector<KeyValue>> matrix;
    matrix.Resize(num_of_el);

    // making counting - filling matrix B
    for(long long  i = 0; i < num_of_el; ++i){
        long double tmp = (long double)input[i].m_key / ((long double)ULLONG_MAX + 10);
        if(matrix[(long long)(( num_of_el * tmp ))].Push_back(input[i]) == false){
            std::cout << "ERROR push_back" << std::endl;
            return 0;
        }
    }   
    
	//sorting every vector in matrix 

    for(long long i = 0; i < matrix.Size(); i++){
        if(matrix[i].Size() != 0) {
            for(long long k = 1; k  < matrix[i].Size(); k++){
            KeyValue tmp;
            tmp.m_key = matrix[i][k].m_key;
            tmp.m_value = matrix[i][k].m_value;
            long long j = k-1;
                while(j >= 0 && matrix[i][j].m_key > tmp.m_key){
                    matrix[i][j+1].m_key = matrix[i][j].m_key;
                    matrix[i][j+1].m_value = matrix[i][j].m_value;
                    matrix[i][j].m_key = tmp.m_key;
                    matrix[i][j].m_value = tmp.m_value;
                    j--;
                }
            }
        }
    }
    
	//printing result 
    for(unsigned long long i = 0; i < num_of_el; i++){
        if(matrix[i].Size() != 0){
            for(unsigned long long j = 0; j < matrix[i].Size(); j++){
                std::cout << matrix[i][j].m_key << "\t" << matrix[i][j].m_value << std::endl;
            }
        }
    }
    
}
\end{lstlisting}


\pagebreak

\section{Консоль}
\begin{alltt}

denis@denis-Extensa-2510G:~/labs_3sem/labs_da/1lab_debug/testing\$ sudo ./wrapper.sh 
Execute tests/01.t
OK
Execute tests/02.t
OK
Execute tests/03.t
OK
\end{alltt}

\textbf{Пример теста 01.t:}

3689348814741910528	vKLwRKbNx

14757395258967642112	eaTYYzeyW

11068046444225732608	oXVGXjwKN

0	aBlXfuZsC

7378697629483821056	XFgYykwmA




\textbf{Полученный ответ 01.a:}

0	aBlXfuZsC

3689348814741910528	vKLwRKbNx

7378697629483821056	XFgYykwmA

11068046444225732608	oXVGXjwKN

14757395258967642112	eaTYYzeyW

\pagebreak

