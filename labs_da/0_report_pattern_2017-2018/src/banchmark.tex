\section{Тест производительности}
\begin{lstlisting}[language=Python]
import random
import string
from decimal import Decimal

MAX_KEY_VAL = 18446744073709551615
MAX_VAL_LEN = 10

def generate_random_value():
    return "".join([random.choice(string.ascii_letters) for _ in range(1, MAX_VAL_LEN)])

if __name__ == "__main__":
    for num in range(1, 2):
        values = list()
        output_filename = "tests/{:02d}.t".format(num)
        N = random.randint(5, 6)
        keyadd = MAX_KEY_VAL/N
        key = 0
        for _ in range( N ):
            value = generate_random_value()
            values.append( (key, value) )
            key = key + keyadd
	
        random.shuffle(values)

        with open( output_filename, 'w') as output:
            for item in values:
                output.write( "{0:.0f}\t{1:}\n".format( item[0], item[1] ) )
        
        output_filename = "tests/{:02d}.a".format( num )
        with open( output_filename, 'w') as output:
            values = sorted( values, key=lambda x: x[0] )
            for value in values:
                output.write( "{0:.0f}\t{1:}\n".format(value[0], value[1]) )   
\end{lstlisting}
Первый результат - результат карманной сортировки. Второй - сортировки вставкой.

\begin{center}
\begin{table}[h]

\label{tabular:timesandtenses}

\begin{tabular}{c|c|c}
Кол-во тестов & Bucket Sort (s) & Insert Sort (s)\\

100 & 6.999999999999999895e-06
 & 0.00023399999999999999576\\
10000 & 0.0004550000000000000008 & 2.1180940000000001433\\
40000 & 0.0017530000000000000009 & 33.79475999999999658\\
\end{tabular}
\end{table}
\end{center}
Как мы видим, на равномерно распределенных последовательностях наша сортировка работает гораздо быстрее.


 
\pagebreak
